\input chr.tex


\makeatletter
\def\tuplehelper#1:#2\@nil{\def\fst{#1}\def\snd{#2}}
\def\tuple#1{\edef\@tmp{#1}\expandafter\tuplehelper\@tmp\@nil}
% plain tex has no default add
\def\add#1#2{\chr@tempcount=#1\relax\advance\chr@tempcount by #2\relax\edef\res{\the\chr@tempcount}}
\makeatother

\chr{fib test}{%
   \enablelog
   \LimitCycles{25}
   \Compose{
      \Rule{main}%
      {} % empty kept head
      % \chr@log{::::::: \fst,\snd}
      {{\tuple{\c}\ifnum\fst>0}} % removed head
      {\true}
      {% poor man's tuple
         \tuple{\c{0}}%
         \add{\fst}{\snd}%
         % ebody expands its argument to replace
         \ebody{\snd:\res}%
      }
   }
   % here we want to show, that we can ignore constraints :D
   \Run{{0:1},{1:1},{-1:1}}
   % print it to file :3
   \printlist{chr@constraints}%
}
\bye