% how to do mat(s)chings!
% #1 is the name of the list of patterns, #2 the name of the list of constraints
%
\def\chr@match#1#2{%
   % length of patterns
   \edef\chr@patterns@length{\csname #1length\endcsname}%
   \edef\chr@match@result{}%
   \def\chr@match@patternnames{#1}%
   \def\chr@match@constraintnames{#2}%
   \let\chr@@output\chr@match@output
   \chr@permute{#2}%
   \chr@log{XX: \chr@match@result}%
}

\def\chr@match@output{\bgroup
   % for list name '#2', try each pattern in list '#1' respectively and return true if all do, return false otherwise
   \chr@tempcountc=\z@
   \let\chr@match@stop=\chr@ns
   \gdef\chr@match@output{}%
   \loop
      \ifnum\the\chr@tempcountc<\chr@patterns@length\relax
         % DO THE WORK
         % now check, if the predicate matches, first we store the current constraint in \c
         \chr@list@let\c=\chr@match@constraintnames[\the\chr@tempcountc]%
         % TODO: only works, if constraint can be printed!
         \chr@verbose{Checking pattern \the\chr@tempcountc\space for constraint value: '\c'.}%
         \let\chr@pattern@result=\chr@empty
         %
         % now we can check the predicate
         \chr@exec{\chr@match@patternnames}%
         \chr@step\chr@tempcountc
      \else
         \let\chr@match@stop=\chr@empty
      \fi
      %
      \ifx\chr@match@stop\chr@ns
   \repeat
   % if it is successful, store it!
   \ifx\chr@pattern@result\chr@ns
      \chr@verbose{Adding successful: \chr@match@output}%
      \chr@list@to@string\chr@list@out{\chr@match@constraintnames}{\chr@patterns@length}% the permutation:
      \expandafter\chr@gapp\expandafter\chr@match@result\expandafter{\expandafter{\chr@list@out},}%
      \chr@verbose{Result: \chr@match@result}%
   \fi
\egroup
}


\def\chr@exec#1{% we have to hide the \else from early execute
   \csname #1@\the\chr@tempcountc\endcsname
      \let\chr@pattern@result=\chr@ns
      \chr@verbose{Pattern \the\chr@tempcountc\space matches.}%
      \relax % relax, we are good!
   \else
      \chr@verbose{Pattern \the\chr@tempcountc\space does not match.}%
      \let\chr@match@stop=\chr@empty
   \fi
}