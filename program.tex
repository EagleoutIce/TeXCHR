% to start every program, allows to define a rule composition with a name (#1) and rule-definitions (#2)
% afterwards you get \#1 to run the program with a constraint list name
% note: your name (#1) must be a valid command sequence to work!
\long\def\chr@program#1#2{\begingroup
   \edef\chr@current{#1}%
   \let\makelist=\chr@list
   \let\listequal=\chr@list@equal
   \let\compose=\chr@compose
   \let\permute=\chr@permute
   \let\match=\chr@match
   \let\rule=\chr@rule
   \let\body=\chr@body
   \let\ebody=\chr@ebody
   \let\run=\chr@run
   % TODO: find better name :D
   \let\Rule=\chr@rule@inline
   \let\Compose=\chr@compose@inline
   \let\Run=\chr@run@inline
   \let\LimitCycles=\chr@limitcycles
   \let\defsolvers=\chr@program@solvers
   \chr@list{chr@constraints}{}%
   % list of rules for the current program
   \chr@list{chr@rules}{}%
   \ignorespaces#2\relax
   \endgroup
}

\def\chr@limitcycles#1{%
   \def\chr@maxruncycles{#1}%
}

% program-scoped setter and getter
\def\chr@program@set#1{\expandafter\def\csname chr@\chr@current @#1\endcsname}
\def\chr@program@get#1{\csname chr@\chr@current @#1\endcsname}

\def\chr@program@solvers#1{%
   \chr@list{\chr@current @solvers}{#1}%
   \def\solvers{\csname\chr@current @solvers\endcsname}%
}

\let\chr\chr@program