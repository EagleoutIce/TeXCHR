% #1 macro name (allows quicker access), #2 comma-separated elements
\def\chr@list#1#2{%
   \chr@tempcount=\z@
   \chr@for\chr@elem:={#2}\do{%
      % you gotta love tex :3 we have to expand \chr@elem once to persist the content
      \expandafter\expandafter\expandafter\def
         \expandafter\csname\expandafter#1@\expandafter\the\expandafter\chr@tempcount
         \expandafter\endcsname\expandafter{\chr@elem}%
      \chr@step\chr@tempcount
   }%
   \expandafter\def\csname#1\endcsname##1{\csname#1@##1\endcsname}%
   \expandafter\edef\csname#1length\endcsname{\the\chr@tempcount}%
}

% #1 name of list, #2 element to append
\def\chr@list@append#1#2{%
   \chr@tempcount=\csname#1length\endcsname
   \chr@step\chr@tempcount
   \expandafter\def\csname#1@\the\chr@tempcount\endcsname{#2}%
   \expandafter\edef\csname#1length\endcsname{\the\chr@tempcount}%
}
% #1 name of list, #2 index to delete
\def\chr@list@delete#1#2{%
   % TODO: check for if last element => pop
   % otherwise replace with \main{next-idx}
}

\chr@list{main}{4,2,3}
\chr@log{\main{1}}
\chr@log{\mainlength}
\chr@list@append{main}{5}
\chr@log{\main{4}}
\chr@log{\mainlength}
