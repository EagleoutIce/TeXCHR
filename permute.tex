% we use heap's algorithm to generate all permutations of a list
% to avoid deep expansions for larger lists, i use the non-recursive variant
% #1 the name of the list
\def\chr@permute#1{%
   \let\chr@@output\chr@list@log
   \edef\chr@tmp{\csname #1length\endcsname}%
   \expandafter\chr@permute@generate\expandafter{\chr@tmp}{#1}%
   % TODO: skip empty on retrieval
}

\def\chr@permute@init#1{%
   \chr@verbose{Initialize array C.}%
   \def\chr@tmp{0}
   \ifnum#1>1\relax
      \chr@tempcount=\z@
      \chr@step\chr@tempcount % we step early to skip the 0 :3
      \loop
         \chr@app\chr@tmp{,0}%
         \chr@step\chr@tempcount
         \ifnum\the\chr@tempcount<#1\relax
      \repeat
   \fi
   % c allows us to use expandafter more easily and is consistent with the pseudo-code
   \expandafter\chr@list\expandafter{\expandafter C\expandafter}\expandafter{\chr@tmp}
}
% #1: n of iterations, #2: list name
% as a slight modification we call \chr@@output for 'output'
\def\chr@permute@generate#1#2{\begingroup
   \chr@verbose{Generate all permutations of list #2 (length: #1).}%
   \ifnum#1=\z@
      \gdef\chr@return{}%
   \else
      % initialize new array - also known as the ""list""
      \chr@permute@init{#1}%
      \chr@@output{#2}%
      % tempcount will be our i
      \chr@tempcount=1\relax
      \let\chr@permute@done\chr@empty
      \loop
         \ifnum\the\chr@tempcount<#1\relax % essentially the head/while condition
            \edef\chr@c@i{\C{\the\chr@tempcount}} % number, we now we can edef C[i]
            \ifnum\chr@c@i<\the\chr@tempcount\relax
               \ifodd\the\chr@tempcount\relax% careful, we have to swap here, as tex presents \ifodd, but the default implementation uses "\ifeven"
                  \chr@list@swap{#2}{\chr@c@i}{\the\chr@tempcount}% swap(A[C[i]], A[i])
               \else
                  \chr@list@swap{#2}{0}{\the\chr@tempcount}% swap(A[0], A[i])
               \fi
               \chr@@output{#2}%
               % without numexpr this is like... really painful
               \chr@tempcountb=\chr@c@i\relax
               \chr@step\chr@tempcountb
               \chr@list@eset C[\the\chr@tempcount]=\the\chr@tempcountb;% C[i] := C[i] + 1
               \chr@tempcount=1\relax % i := 1
            \else
               \chr@list@set C[\the\chr@tempcount]=0;% C[i] := 0
               \chr@step\chr@tempcount % i++
            \fi
         \else
            % while loop is done!
            \let\chr@permute@done\chr@ns
         \fi
         \ifx\chr@permute@done\chr@empty
      \repeat
   \fi
   \endgroup
}