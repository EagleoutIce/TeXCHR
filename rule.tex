% #1: kept_head
% #2: removed_head
% #3: guard
% #4: body
% #5: name (afterthought)
% returns a solver (stored in \chr@toreturn)
\def\chr@rule#1#2#3#4#5{%
   % takes the name of the constraint list to work on
   \chr@program@set{chr@toreturn}##1{%
      % first, let's match
      \chr@list@concat{chr@all}{#1}{#2}% we want k + r
      \chr@match{chr@all}{##1}%
      \chr@log{Matched: \chr@match@result}%
      % iterate and check if the guards like them.
      % we pass in the list getter to the guard as \cs, so it can use \cs1, \cs2, ...
      \let\chr@constraints@to@fire=\chr@empty
      % todo: just iter directly?
      \chr@list{chr@guard@tries}{\chr@match@result}%
      \chr@list@iter\chr@try:=chr@guard@tries\do{%
         \expandafter\chr@list\expandafter{\expandafter c\expandafter s\expandafter}\expandafter{\chr@try}%
         \chr@exec@guard{#3}%
             {\global\let\chr@constraints@to@fire=\chr@try\relax \chr@break}%
             {}% just continue in the other case
      }%
      \ifx\chr@constraints@to@fire\chr@empty
         \chr@verbose{No guard matched.}%
         \chr@toreturn{}%
      \else % here, we have a matched head and guard, and we can actually apply the rule!
         % we have the length of the kept head in #1, now we shorten the list to only contain it!
         \chr@log{Firing rule '#5' with constraints: \chr@constraints@to@fire}%
         % we copy this way, ugly? yes. nice? i dunno
         \chr@list{chr@list@tmp}{\chr@constraints@to@fire}%
         % TODO: we have to remove the _unique_ respective elements from the constraints? first?
         \chr@list@to@string\chr@constraints@remains{chr@list@tmp}{\csname#1length\endcsname}%
         \chr@list{chr@c@rem}{\chr@constraints@remains}% these are those to remove
         \chr@list@iter\chr@elem:=chr@c@rem\do{%
            \chr@list@remove{chr@constraints}{\chr@elem}% these are those to keep
         }%
         % the body should return a new list named... result! which we now simply concat, it can still access c for a new list
         \let\c=\chr@constraints
         #4
         \chr@list@concat{chr@tmp}{chr@constraints}{result}%
         \chr@list@concat{chr@constraints}{chr@tmp}{@empty@list}%
      \fi
   }
}

\def\chr@body#1{%
   \chr@list{result}{#1}%
}
\def\chr@ebody#1{%
   \edef\chr@tmp{#1}\def\chr@listname{result}%
   \chr@list{result}{}%
   \expandafter\chr@list@append\expandafter{\expandafter\chr@listname\expandafter}\expandafter{\chr@tmp}%
}

% variant of \chr@rule, which allows for a name as #1 (which must be unique)
% and uses it to construct lists in-place
\def\chr@rule@inline#1#2#3#4#5{%
   \chr@list{#1@kept}{#2}%
   \chr@list{#1@rem}{#3}%
   \chr@rule{#1@kept}{#1@rem}{#4}{#5}{#1}%
}

% #2 if matches, #3 otherwise
\def\chr@exec@guard#1{% we have to hide the \else from early execute
   #1
      \let\chr@guard@result=\chr@ns
      \chr@verbose{Guard matches.}%
      \expandafter\chr@firstoftwo
   \else
      \chr@verbose{Guard does not match.}%
      \expandafter\chr@secondoftwo
   \fi
}